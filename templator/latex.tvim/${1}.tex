\documentclass{article}
<?vimcursor?>
% ==============================
% Constants
\newcommand{\MyDocTitle}{Übung 1}
\newcommand{\MyDocAuthor}{David Holland - \textit{Übungsgruppe}}
\newcommand{\MyDocDate}{\today{}}
% END Constants
% ==============================

\usepackage{techmath}

% ==============================
% Meta
\title{\MyDocTitle{}}
\author{\MyDocAuthor{}}
\date{\MyDocDate{}}
% END Meta
% ==============================


\begin{document}

\renewcommand{\seriesname}{Aufgabe}
\setcounter{exerciseseries}{0}

\maketitle

\tableofcontents

\newpage

% ==============================
% Beginning of Document
%
% Note: For subfiles, use
%
%       \subfile{filename.tex}
%
%       and in the subfile:
%
%       \documentclass[main_filename.tex]{subfiles}
%
%       and leave the whole
%       preamble out.
%       Also don't forget to insert
%       \begin{document} and
%       \end{document} into subfiles
% ==============================

\section{Hausübung}

\subsection{Aufgabe H1 {\normalfont (Aufgabentitel)}}

\subsubsection{Aufgabenstellung}
\begin{enumerate}[label=(\alph*)]
	\item Aufgabentext
		\begin{enumerate}[label=(\roman*)]
			\item Teilaufgabe
		\end{enumerate}
\end{enumerate}

\subsubsection{Lösung}
\begin{enumerate}[label=(\alph*)]
	\item Aufgabentext
		\begin{enumerate}[label=(\roman*)]
			\item Lösung Teilaufgabe
		\end{enumerate}
\end{enumerate}

\end{document}
